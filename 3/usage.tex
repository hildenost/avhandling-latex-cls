\chapter{Class usage}

\intro{How to tweak the look of this very chapter introduction/abstract, and how to turn off
the chapter table of contents.
Also, a note on the todo-notes is added.}

\section{Chapter abstract and table of contents}

I created the command \lstinline{\intro} for the special formatting of the chapter abstract. It is defined in \texttt{avhandling.cls}, and can of course be tweaked to your liking.

In addition, I added a so-called \emph{minitoc} at the end of this chapter abstract.
If you'd like to keep the abstract, but get rid of the minitoc, just delete the \lstinline{\chaptertoc} in the \lstinline{\intro} definition in the \texttt{avhandling.cls} file.

The depth of the table of contents is set to 1, that is, just the section headings are included. Changing \textbf{tocdepth} to 2 will include subsections as well.

\section{Todo-notes}

During the writing process, I used todo-notes in the margin to help keep track of comments\todo{Like so}\todofig.
The package I use, and which is included in this class, is the \textbf{fixme} package\todocite.
I redefined the commands to be more descriptive of how I would use the notes.

There are 3 commands: \lstinline{\todo}, \lstinline{\todocite}, and \lstinline{\todofig}.
The \lstinline{\todo} command needs text input in curly braces, like so: \lstinline|\todo\{Remember this!}|\todocite[Perhaps cite the package manual?].
The other two add by default the text of "citation needed" and "figure wanted".
If you want to describe the problem more, just write of your heart's content in square brackets like so: \lstinline{\todocite[A very very important paper by a very very important author should deffo be cited here.]}. 
